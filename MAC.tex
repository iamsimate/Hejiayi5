\documentclass{article}
\usepackage{listings}
\lstset{language=Matlab,frame=shadowbox,
	rulesepcolor=\color{red!20!green!20!blue!20},
	keywordstyle=\color{blue!90}\bfseries,
	showstringspaces=false,
	numbers=left,
	numberstyle=\tiny,
	stringstyle=\ttfamily,
	breaklines=true,
	extendedchars=false,}%代码语言使用的是matlab
\usepackage{xcolor}
\usepackage{color}
\definecolor{dkgreen}{rgb}{0,0.6,0}
\usepackage{ctex}
\usepackage{amssymb}\usepackage[a4paper, body={18cm,22cm}]{geometry}
\usepackage{amsmath,amssymb,amstext,wasysym,enumerate,graphicx}
\usepackage{float,abstract,booktabs,indentfirst,amsmath}
\usepackage{amsthm, bm, hyperref, mathrsfs, tikz, color, framed}
\usepackage{mathrsfs}
\usepackage{array}
\usepackage{booktabs} %调整表格线与上下内容的间隔
\usepackage{multirow}
\usepackage{diagbox}
\renewcommand\arraystretch{1.4}
\usepackage{indentfirst}
\setlength{\parindent}{2em}

\geometry{left=2.8cm,right=2.2cm,top=2.5cm,bottom=2.5cm}
%\geometry{left=3.18cm,right=3.18cm,top=2.54cm,bottom=2.54cm}

\graphicspath{{figures/}}

\title{\heiti 《微分方程数值解实验报告》 }
\newcounter{theoremname}
\newenvironment{theorem}{\begin{shaded}\stepcounter{theoremname}\par\noindent\textbf{定理\arabic{theoremname}. }}{\end{shaded}\par}
\definecolor{shadecolor}{RGB}{241, 241, 255}
\newcounter{definitionname}
\newenvironment{definition}{\begin{shaded}\stepcounter{definitionname}\par\noindent\textbf{定义\arabic{definitionname}. }}{\end{shaded}\par}
\newcounter{problemname}
\newenvironment{problem}{\begin{shaded}\stepcounter{problemname}\par\noindent\textbf{题目\arabic{problemname}. }}{\end{shaded}\par}
\newenvironment{solution}{\par\noindent\textbf{解答. }}{\par}
\newenvironment{note}{\par\noindent\textbf{注记. }}{\par}

\begin{document}
	
	\maketitle
	
	\vspace{5cm}
	
	
	\begin{table}[h]
		\centering
		\begin{Large}
			\begin{tabular}{p{3cm} p{7cm}<{\centering}}
				学  \qquad  校: &  华中科技大学     \\ \cline{2-2}
				学 \qquad 院:      & 数学与统计学院   \\ \cline{2-2}
				成  \qquad  员: & 何嘉怡 \\ \cline{2-2}
				学\qquad 号: &U201916521 \\ \cline{2-2}
				专\qquad 业:&信息与计算数学\\ \cline{2-2}
				指导教师:       & 高华东 \\ \cline{2-2}
			\end{tabular}
		\end{Large}		
	\end{table}
	
	\begin{table}[h]
		
		\setlength{\tabcolsep}{3pt}
		\begin{tabular}{p{2cm} p{5cm}<{\centering}}
			评\qquad 分:&  \\ \cline{2-1}
		\end{tabular}
	\end{table}
	\newpage
	
	\tableofcontents
	\newpage

\section{实验介绍}
\begin{problem}
	
	使用MAC格式计算方腔流(Lid-driven cavity model),要求网格h$<\frac{1}{128}$
	\begin{equation*}
		\left\{\begin{matrix}
			-\Delta \textbf{u}+\bigtriangledown p=0\\ 
			div \textbf{u}=0
		\end{matrix}\right.
	\end{equation*}
这里$\Omega =(0,1)^2$,在上边界,$\textbf{u}=(1,0)^T$,在剩余边界$\textbf{u}=\textbf{0}$
	\end{problem}

\section{实验方法}
\begin{solution}	 
令$\textbf{u}=\binom{u}{v}$

则方程化成\begin{equation*}
	\left\{\begin{matrix}
		-\Delta u+P_x=0\\ 
		-\Delta v +P_y=0\\ 
		u_x+v_y=0
	\end{matrix}\right.
\end{equation*}

目标是将问题转化成\begin{equation*}
	\begin{bmatrix}
		K_1 &O  & B_1\\ 
		O& K_2 &B_2 \\ 
		A& B & O
	\end{bmatrix}
	\begin{bmatrix}
		U\\ 
		V\\ 
		P
	\end{bmatrix}=\beta
\end{equation*}
\begin{figure}[htbp]
	\centering
	\begin{minipage}{0.49\linewidth}
		\centering
		\includegraphics[width=0.9\linewidth]{2.png}
	\end{minipage}
	%\qquad
	\begin{minipage}{0.49\linewidth}
		\centering
		\includegraphics[width=0.9\linewidth]{4.png}		
	\end{minipage}
\end{figure}

其中灰色的为虚拟网格,利用双侧差分格式处理本质边界条件。

交错网格上,$\times$表示p所取格点,$\bullet$表示u所取格点,$\circ$表示v所取格点。最后求解出所有格点后,我们还需将u和v的相邻格点值作平均,来和p格点的坐标对齐。

具体的细节格点如下图:
\begin{figure}[htbp]
	\centering
	\begin{minipage}{0.49\linewidth}
		\centering
		\includegraphics[width=0.9\linewidth]{1.png}
	\end{minipage}
	%\qquad
	\begin{minipage}{0.49\linewidth}
		\centering
		\includegraphics[width=1.0\linewidth]{3.png}		
	\end{minipage}
\end{figure}


因此
\begin{align*}
	U &= \left(u_{11},u_{12},\cdots,u_{1N},\cdots,u_{N-1,1},\cdots,u_{N-1,N}\right)^T\\ 
	V &= \left(v_{11},v_{12},\cdots,v_{1,N-1},\cdots,v_{N,1},\cdots,v_{N,N-1}\right)^T\\ 
	P &= \left(p_{11},p_{12},\cdots,p_{1N},\cdots,p_{N,1},\cdots,p_{N,N}\right)^T
\end{align*}

内部点(中心差分格式)
\begin{equation*}
	\left\{\begin{matrix}
		\frac{p_{i+1,j}-p_{i,j}}{h}-\frac{u_{i+1,j}+u_{i-1,j}+u_{i,j+1}+u_{i,j-1}-4u_{i,j}}{h^2}=0\\ 
		\frac{p_{i,j+1}-p_{i,j}}{h}-\frac{v_{i+1,j}+v_{i-1,j}+v_{i,j+1}+v_{i,j-1}-4v_{i,j}}{h^2}=0
	\end{matrix}\right.
\end{equation*}

处理边界条件
\begin{equation*}
	\left\{\begin{matrix}
		\frac{u_{i,N}+u_{i,N+1}}{2}=1\\ 
		\frac{p_{i+1,N}-p_{i,N}}{h}-\frac{u_{i+1,N}+u_{i-1,N}+u_{i,N+1}+u_{i,N-1}-4u_{i,N}}{h^2}=0
	\end{matrix}\right.
\end{equation*}

得\begin{equation*}
	\frac{p_{i+1,N}-p_{i,N}}{h}-\frac{u_{i+1,N}+u_{i-1,N}+u_{i,N-1}-5u_{i,N}}{h^2}=\frac{2}{h^2}
\end{equation*}
\end{solution}

同理得
\begin{equation*}
	\frac{p_{N,j+1}-p_{N,j}}{h}-\frac{v_{N-1,j}+v_{N,j+1}+u_{N,j-1}-5u_{N,j}}{h^2}=0
\end{equation*}

从而得矩阵

\begin{equation*}
K_1=\frac{1}{h^2}	\begin{bmatrix}
		S& Q &  & \\ 
		Q& S & \ddots & \\ 
		& \ddots &\ddots  &Q \\ 
		&  & Q & S
	\end{bmatrix}_{(N-1)N\times (N-1)N}
\end{equation*}
其中
\begin{equation*}
	S=\begin{bmatrix}
		5 &  -1&  &  & \\ 
		-1&  4&-1  &  & \\ 
		&\ddots & \ddots &\ddots  & \\ 
		&  & -1& 4 &-1 \\ 
		&  &  &  -1& 5
	\end{bmatrix}_{N\times N}Q=\begin{bmatrix}
	-1&  &  &  & \\ 
	&  -1&  &  & \\ 
	& & \ddots &  & \\ 
	&  & & -1 & \\ 
	&  &  &  & -1
\end{bmatrix}_{N\times N}
\end{equation*}
同理
\begin{equation*}
	K_2=\frac{1}{h^2}	\begin{bmatrix}
		S_1& Q_1 &  & \\ 
		Q_1& S_1 & \ddots & \\ 
		& \ddots &\ddots  &Q_1 \\ 
		&  & Q_1 & S_1
	\end{bmatrix}_{(N-1)N\times (N-1)N}
\end{equation*}
其中
\begin{equation*}
	S_1=\begin{bmatrix}
		5 &  -1&  &  & \\ 
		-1&  4&-1  &  & \\ 
		&\ddots & \ddots &\ddots  & \\ 
		&  & -1& 4 &-1 \\ 
		&  &  &  -1& 5
	\end{bmatrix}_{N-1\times N-1}	Q1=\begin{bmatrix}
	-1&  &  &  & \\ 
	&  -1&  &  & \\ 
	& & \ddots &  & \\ 
	&  & & -1 & \\ 
	&  &  &  & -1
\end{bmatrix}_{N-1\times N-1}
\end{equation*}

同时\begin{equation*}
	A=\frac{1}{h}	\begin{bmatrix}
		S_2&  &  & \\ 
		Q_2& S_2 &  & \\ 
		& \ddots &\ddots & \\ 
		& & & S_2&\\
		&  & & Q_2
	\end{bmatrix}_{N^2\times (N-1)N}
\end{equation*}
其中\begin{equation*}
S_2=-\begin{bmatrix}
	-1&  &  &  & \\ 
	&  -1&  &  & \\ 
	& & \ddots &  & \\ 
	&  & & -1 & \\ 
	&  &  &  & -1
\end{bmatrix}_{N\times N}	Q_2=\begin{bmatrix}
-1&  &  &  & \\ 
&  -1&  &  & \\ 
& & \ddots &  & \\ 
&  & & -1 & \\ 
&  &  &  & -1
\end{bmatrix}_{N\times N}
\end{equation*}

\begin{equation*}
	B=\frac{1}{h}	\begin{bmatrix}
		S_3&  &  & \\ 
		& S_3 &  & \\ 
		&  &\ddots  & \\ 
		&  &  & S_3
	\end{bmatrix}_{N^2\times (N-1)N}
\end{equation*}
其中\begin{equation*}
	S_3=\begin{bmatrix}
		-1&  &  &  & \\ 
		1&  -1&  &  & \\ 
		& \ddots& \ddots &  & \\ 
		&  & 1& -1 & \\ 
		&  &  & 1 & -1
	\end{bmatrix}_{N\times N-1}
\end{equation*}
同时
\begin{equation*}
	B_1=\frac{1}{h}\begin{bmatrix}
		S_2 &S_2  &  &  & \\ 
		&  S_2& S_2 &  & \\ 
		&  &  \ddots& \ddots & \\ 
		&  &  & S_2 & S_2
	\end{bmatrix}_{(N-1)N\times N^2}
\end{equation*}

\begin{equation*}
B_2=\frac{1}{h}	\begin{bmatrix}
		S_3^T &  &  & \\ 
		&  S_3^T& & \\ 
		&  &  \ddots& \\ 
		& & &S_3^T
	\end{bmatrix}_{(N-1)N\times N^2}
\end{equation*}

同时,我们需要进行压力矫正,即赋予压力一个初值,把A和B矩阵第一行变为0,且\begin{equation*}
	K(2*(N-1)*N+1,2*(N-1)*N+1)=1
\end{equation*}

最后在计算得到了U、V、P之后,为了画图,我们需要把列矩阵转化成$N \times N$的矩阵,先把U和V的相邻格点作平均,以此来和P的格点坐标对应。

\section{实验结果}
以下为速度u、v、压力p的一些图像,其中压力的图像不太明显,左上角和右上角颜色较深一些。

在实验过程中,想到矩阵的大概形式不难,但是要弄清楚各个分块矩阵的阶数比较不容易。一开始跑出来的结果不正确,是因为我把$K_2$的矩阵写错了,A和B矩阵的阶数弄错,导致方程求解过程中出现了奇异值。
\begin{figure}[htbp]
	\centering
	\begin{minipage}{0.49\linewidth}
		\centering
		\includegraphics[width=0.9\linewidth]{5.png}
		\caption{流线图}
	\end{minipage}
	%\qquad
	\begin{minipage}{0.49\linewidth}
		\centering
		\includegraphics[width=0.9\linewidth]{10.png}	
		\caption{压力的棋盘图}	
	\end{minipage}
\end{figure}
\begin{figure}[htbp]
	\centering
	\begin{minipage}{0.49\linewidth}
		\centering
		\includegraphics[width=0.9\linewidth]{6.png}
		\caption{u的棋盘图}
	\end{minipage}
	%\qquad
	\begin{minipage}{0.49\linewidth}
		\centering
		\includegraphics[width=0.9\linewidth]{8.png}
		\caption{速度u的等值线}		
	\end{minipage}
\end{figure}
\begin{figure}[htbp]
	\centering
	\begin{minipage}{0.49\linewidth}
		\centering
		\includegraphics[width=0.9\linewidth]{7.png}
		\caption{v的棋盘图}
	\end{minipage}
	%\qquad
	\begin{minipage}{0.49\linewidth}
		\centering
		\includegraphics[width=0.9\linewidth]{9.png}
		\caption{速度v的等值线}		
	\end{minipage}
\end{figure}

\section{实验代码}
\begin{lstlisting}[language=matlab]
clc; clear; format longE;

h  =                  1/128; % mesh size in [0,1]
xh =              (0:h:1)'; % nodes in [0,1]
N  =          length(xh)-1; % mesh number in [0,1]


Ahv  = gallery('tridiag', N-1,-1,2,-1);%v
Ahu  = gallery('tridiag', N ,-1,2,-1);%u

I1   = speye(N  ); % some identity matrix
I2  = speye(N-1 ); % some identity matrix
I3=sparse(N-1,N);
I4=sparse(N-1,N-1);

K=sparse(2*(N-1)*N+N^2,2*(N-1)*N+N^2);

u1=sparse(2*(N-1)*N+N^2,1);%处理右端项
for i =1:N-1
ul(i*N)=2/(h^2); 
end


ul(2*(N-1)*N+1,1)=0.005;%压力矫正,%为p赋一个初值0.005

for i=1:N-1
I3(i,i)=-1;
I3(i,i+1)=1;
end

for i=1:N-2
I4(i,i)=-1;    
I4(i,i+1)=1;
end
I4(N-1,N-1)=-1;

K1 = kron(I2, Ahu) + kron(Ahv, I1); % difference matrix for u
K2 = kron(I1 ,Ahv) + kron(Ahu, I2); % difference matrix for v

for i =1:N-1
K1(i*N,i*N)=5; 
K1((i-1)*N+1,(i-1)*N+1)=5;
K2(i*(N-1),i*(N-1))=5;
K2((i-1)*(N-1)+1,(i-1)*(N-1)+1)=5;
end


B1=kron(I3,I1);
B2=kron(I1,I3);


P1=gallery('tridiag',N,-1,1,0);
A=kron(P1,I1);
A=A(:,1:(N-1)*N);


P2=gallery('tridiag',N,0,-1,1);
P2=P2(1:end-1,:);
P3=-P2';
B=kron(I1,P3);

K(1:(N-1)*N,1:(N-1)*N)=K1./(h^2);
K((N-1)*N+1:2*(N-1)*N,(N-1)*N+1:2*(N-1)*N)=K2./(h^2);
K(1:(N-1)*N,2*(N-1)*N+1:end)=B1./h;
K((N-1)*N+1:2*(N-1)*N,2*(N-1)*N+1:end)=B2./h;
K(2*(N-1)*N+1:end,1:(N-1)*N)=A./h;
K(2*(N-1)*N+1:end,(N-1)*N+1:2*(N-1)*N)=B./h;

K(2*(N-1)*N+1,:)=0;
K(2*(N-1)*N+1,2*(N-1)*N+1)=1;%给压力P赋一个初值

%用matlab自带的求解方程
uh = K\ul; 

uhu=uh(1:(N-1)*N);%u值
uhv=uh(N*(N-1)+1:2*N*(N-1));%v值
uhp=uh(2*(N-1)*N+1:2*(N-1)*N+N^2);%p值

uu=zeros(N,N+1);%存储u网格值,包括左右边界
uu(:,2:N)=reshape(uhu,N,N-1);
uv=zeros(N+1,N);%存储v网格值,包括上下边界
uv(2:N,:)=reshape(uhv,N-1,N);
up=reshape(uhp,N,N);%存储p网格值

%利用平均,把u,v的点和p点对齐
uu1=zeros(N);
uv1=zeros(N);
for i=1:N
uu1(:,i)=(uu(:,i)+uu(:,i+1))./2;
uv1(i,:)=(uv(i,:)+uv(i+1,:))./2;
end

%画图
x=[h/2:h:1-h/2];
y=[h/2:h:1-h/2];
[X,Y]=meshgrid(x,y);

figure(1)
pcolor(X,Y,up);
figure(2)
pcolor(X,Y,uu1);
figure(3)
pcolor(X,Y,uv1);
figure(4)
contourf(X,Y,uu1);
figure(5)
contourf(X,Y,uv1);
figure(6)
streamslice(uu1,uv1);
\end{lstlisting}
	\end{document}